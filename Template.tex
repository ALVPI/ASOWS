\documentclass[10pt]{article}

% Idioma y codificación
\usepackage[utf8]{inputenc}
\usepackage[spanish]{babel}

% Márgenes
\usepackage{geometry}
\geometry{top=2.5cm, bottom=2.5cm, left=2.5cm, right=2.5cm}

% Gráficos
\usepackage{graphicx}
\usepackage{graphics}
\usepackage{wrapfig}
\usepackage{float}
\usepackage{tabularx}
\usepackage{multirow}
\usepackage{array}
\usepackage{ragged2e}
\usepackage{colortbl}
\usepackage{booktabs}
\usepackage{amssymb}
\usepackage{multicol}
\usepackage[hypcap]{caption}
\usepackage[backend=biber, style=apa]{biblatex} % Cambia 'apa' si usas otro estilo
\addbibresource{bibliografia.bib}

% Listings para código
\usepackage{listings}
\usepackage{xcolor}
\usepackage{caption}
\usepackage{capt-of}
\lstdefinestyle{terminal}{
	backgroundcolor=\color{gray!10},
	basicstyle=\ttfamily\small,
	keywordstyle=\color{blue},
	commentstyle=\color{gray},
	frame=single,
	columns=fullflexible,
	morecomment=[l]{//}
}
\usepackage{listings}
\lstset{
	inputencoding=utf8,
	extendedchars=true,
	literate={á}{{\'a}}1 {é}{{\'e}}1 {í}{{\'i}}1 {ó}{{\'o}}1 {ú}{{\'u}}1 {ñ}{{\~n}}1
}

% Hipervínculos
\usepackage[hidelinks]{hyperref}
\hypersetup{
	colorlinks=true,
	linkcolor=blue,
	filecolor=magenta,
	urlcolor=cyan,
}

% Encabezados y pies de página
\usepackage{fancyhdr}
\pagestyle{fancy}
\fancyhf{}
\fancyfoot[R]{\thepage}

% Formato de títulos
\usepackage{titlesec}
\titleformat{\subsection}{\normalfont\large\bfseries}{\hspace{1em}\thesubsection}{1em}{}
\titleformat{\subsubsection}{\normalfont\normalsize\bfseries}{\hspace{2em}\thesubsubsection}{1em}{}

% Datos de la portada
\title{\textbf{Creación de un directorio activo en un Windows Server2022 }}
\author{\textbf{\small Álvaro López Primo}}
\date{}

\makeatletter
\let\thetitle\@title
\let\theauthor\@author
\let\thedate\@date
\makeatother
\definecolor{win11color}{RGB}{245, 249, 255}    % Azul muy claro
\definecolor{win22color}{RGB}{245, 255, 249}    % Verde muy claro
\definecolor{rowcolor1}{RGB}{248, 248, 248}     % Gris muy claro
\definecolor{rowcolor2}{RGB}{255, 255, 255}     % Blanco
\definecolor{headercolor}{RGB}{240, 240, 240}   % Gris para encabezado
\definecolor{bordercolor}{RGB}{150, 150, 150}   % Color de bordes

\begin{document}

	\begin{titlepage}
		\centering
		\includegraphics[scale=0.3]{Recursos/UVA-3881270087.pdf}\\[0.5cm]
		\rule{\linewidth}{0.2mm}\\[0.4cm]
		{\huge\bfseries \thetitle}\\
		\rule{\linewidth}{0.2mm}\\[1.5cm]
		{\Large\bfseries Ingeniería informática}\\[0.3cm]
		{\Large\bfseries Tecnologías de la información y la comunicación}\\[1cm]
		{\Large \theauthor}\\[1.5cm]
		{\Large \thedate}
	\end{titlepage}
	
	\renewcommand{\contentsname}{Índice}
	\begin{multicols}{2} 
		\tableofcontents
	\end{multicols} 

	% --- FIN DEL ÍNDICE DE TABLAS A DOS COLUMNAS ---
	
	\clearpage
	
	% --- INICIO DEL ÍNDICE DE FIGURAS A DOS COLUMNAS ---
	\begin{multicols}{2}
		\listoffigures
	\end{multicols}
	% --- FIN DEL ÍNDICE DE FIGURAS A DOS COLUMNAS ---
	
	\clearpage
\section{Introducción}
	En este apartado responderemos a algunas preguntas antes de empezar con el proceso de instalación de windows server 2022.
	\subsection{¿Cuáles son las principales diferencias entre Windows server 2022 y un Windows 11?}
		Para respodner a esta pregunta, lo primero que tenemos que tener claro es que el \textbf{kernel} de  ambos productos de software de Microsft es el mismo, pero las tareas para las que están especialidas no son las mismas. 
		
		\begin{center}
			\begin{minipage}{0.4\textwidth}
				\centering
				\includegraphics[width=\linewidth]{Recursos/OIP-2538521300.png}
				\captionof{figure}{Distribución de los anillos dentro de un sistema operativo}
				\label{fig:ejemplo}
			\end{minipage}
			\hfill
			\begin{minipage}{0.55\textwidth}
				Para entender esto bien, tenemos que recordar el concepto de los \underline{anillos dentro de un sistema operativo}, los cuales definen la estrcutura de mismo, agrupando los niveles de mayor a menor privilegio. \\
				El kenel, se encuentra en el anillo cero, es decir aquél más cercano a la CPU, por ende el que más "poder" tiene (en última instancia será el encargado de realizar las operaciones sobre el hardware del dispositivo, mediante interrupciones o diferentes métodos, en función de las necesidades de cada proceso. ) \\ Que compartan el kernel implica que comparten el mismo \textit{ntoskrnl.exe} (en windows este ejecutable es el manejador del kernel),la misma abtracción de hardware (planificación para el \textit{multithreading, manejo de interrupciones\ldots})
			\end{minipage}
		\end{center}
		Ahora bien, si el kernel es el mismo ¿por qué existen dos versiones diferentes de Windows? Esta pregunta se responde de manera bastante fácil, puesto que el motivo es el propósito para el que han sido diseñados estos productos de software por parte de Microsft. El \textit{Windows 11} es un software pensado para clientes, es decir dispositivos o hosts individuales, los cales, presentan un hardware comedido para un uso ofimático o para tareas poco pesadas (computacionalmente hablando), mientras que \textit{Windows Server} está pensado como orquestador, es decir, para ofrecer un servicio a diferentes clientes, los cuales realizaran tareas sobre la máquina que hostea este sisitema operativo, está pensado para gestionar los recursos de la máquina donde ha sido instalado así como para ofrecer servicio a múltiples hosts clientes. Para mejorar la legibilidad de esta comparativa, he decidido crear una tabla donde se muestran las principales diferencias:
	\begin{table}[H]
		\centering
		\renewcommand{\arraystretch}{1.3}
		\rowcolors{2}{rowcolor1}{rowcolor2}
		\arrayrulecolor{bordercolor}
		\setlength{\arrayrulewidth}{0.8pt}
		\begin{tabularx}{\textwidth}{|>{\RaggedRight}p{3.2cm}|>{\RaggedRight\columncolor{win11color}}p{6.5cm}|>{\RaggedRight\columncolor{win22color}}p{6.5cm}|}
			\hline
			\rowcolor{headercolor}
			\textbf{Ámbito} & \textbf{Windows 11 (cliente)} & \textbf{Windows Server 2022} \\
			\hline
			\textbf{Licensing model} & Retail/OEM por dispositivo; sin límite de conexiones & Servidor + CAL (usuario o dispositivo); valida conexiones \\
			\hline
			\textbf{Roles de red} & Ninguno; límite 20 conexiones SMB & AD DS, DNS, DHCP, RRAS, NLB, S2D, Storage Replica… \\
			\hline
			\textbf{Seguridad por defecto} & Secured-core PC optativo; HVCI manual & Secured-core Server habilitado; VBS + DMA Guard \\
			\hline
			\textbf{Hyper-V} & $\leq$ 2 CPU / 2 TB RAM; sin réplica ni Live Migration & 48 TB RAM por VM; réplica, nested, SR-IOV, SET \\
			\hline
			\textbf{Almacenamiento kernel} & NTFS + ReFS deshabilitado arranque; sin dedup & ReFS v3 block-cloning; dedup; S2D; Storage Replica \\
			\hline
			\textbf{Protocolo de ficheros} & SMB 3.1.1 básico & SMB 3.1.1 + SMB over QUIC (UDP-TLS 1.3) \\
			\hline
			\textbf{Quantum planificador} & Corto, prioriza latencia interactiva & Largo, prioriza throughput; timer 100 Hz \\
			\hline
			\textbf{Interfaz por defecto} & Shell moderno, Store, Widgets & Desktop Experience opcional; recomendado Server Core \\
			\hline
			\textbf{Ciclo de patch} & 1 actualización funcional/año (H2) & Canal LTSC; 10 años soporte; parches acumulativos \\
			\hline
			\textbf{Integración cloud} & M365, OneDrive, Teams consumer & Azure Arc, JEA, Windows Admin Center, HCI \\
			\hline
		\end{tabularx}
		\caption{Comparativa entre Windows 11 y Windows Server 2022}
		\label{tab:windows_comparison_vertical}
	\end{table}
\section{Instalación de \textit{Windows server} sobre la máquina virtual}
	En el guión de la práctica se comenta que hay diversas opciones para la instalación del software, las máquinas virtuales propias de la asignatura (mediante matrix) o una versión \textit{offline} que se basa en la instalación del software en una partición o empleando los recursos de mi host.\\ 
	Finalmente me he decantado por la máquina virtual puesto que en mi pc no tengo el espacio suficiente como para poder hacer una partición dedicada para la práctica, por lo que a partir de ahora, todas las capturas de \textit{Windows Server} serán también capturas de la máquina virtual de la asignatura.\\ 
	\subsection{Formateo de la máquina virtual}
	\begin{center}
		\begin{minipage}{0.4\textwidth}
			\centering
			\includegraphics[width=\linewidth]{Recursos/SeleccionISO}
			\captionof{figure}{Seleccionaremos la ISO de Windows Server.}
			\label{seleccionOS}
		\end{minipage}
		\hfill
		\begin{minipage}{0.55\textwidth}
		Anteriormente la máquina virtual de la asignuta ha sido empleda para realizar los guiones de prueba sobre un Ubuntu, por lo que el primer paso es formatear la máquina e instalar \textit{Windows Server}. Como se indica en el guión, lo que haremos será entra en la BIOS de la máquina virtual y en el \textit{Boot Menu} elegir la opción de \textit{UEFI QEMU DVDROM}, con esto lo que haremos será iniciar el instalador de \textit{Windows Server} desde el CD y podremos comenzar con el proceso de instalación.
		\end{minipage}
	\end{center}
	\begin{figure}[H]
		\centering
		\begin{minipage}{0.45\textwidth}
			\centering
			\includegraphics[width=\textwidth]{Recursos/SeleccionVersionWindowsServer.png}
			\caption{Seleccionamos la versión de \textit{Windows Server} que se nos indica en el enunciado}
			\label{versionWS}
		\end{minipage}
		\hfill
		\begin{minipage}{0.45\textwidth}
			\centering
			\includegraphics[width=\textwidth]{Recursos/ProblemaInstalacionWindowsServer.png}
			\caption{Problema con la instalación, no reconoce que haya discos duros en la máquina virtual.}
			\label{ProblemaInstalacionWS}
		\end{minipage}
	\end{figure}

	Una vez que hemos decidido qué versión de \textit{WS} vamos a instalar, y que deseamos borrar todo el disco duro sin mantener particiones o datos de las prácticas previas, lo que nos encontramos es que \textit{no reconoce ninguno disco}, esto no es posible puesto que previamente hemos trabajado con esta máquina virtual y teníamos 60GB de espacio en la misma, por lo que  me decanto con un problema de \textit{drivers} es decir los "traductores" que permiten al OS emplear el hardware del pc. \\
	En el guión de la práctica ya se mencionaba esta posiblidad por lo que realizaremos los pasos que ahí se muestran
	\begin{figure}[H]
		\centering
		\begin{minipage}{0.45\textwidth}
			\centering
			\includegraphics[width=\textwidth]{Recursos/virtioiso.png}
			\caption{Cargamos el disco (en el mismo menú en el que hemos seleccionado la imagen del \textit{WS})}
			\label{CargarElCDDrivers}
		\end{minipage}
		\hfill
		\begin{minipage}{0.45\textwidth}
			\centering
			\includegraphics[width=\textwidth]{Recursos/seleccionDrier.png}
			\caption{Una vez que ya tenemos esto, volvemos a la parte de Console dento de promox y buscamos los drivers para los discos duros}
			\label{SeleccionarDriver}
		\end{minipage}
	\end{figure}
		En las opciones de driver para el disco duro, tenemo diferenes versiones, lo que debemos hacer es seleccionar la que sale seleccionada pueste que se diferencian en el año en el que han sido compilados. En nuestro caso como estamos instalando la versión de 2022, para que el driver de los menores problemas de compatibilidad posibles seleccionamos la \textit{2k22}
			\begin{center}
			\begin{minipage}{0.4\textwidth}
				\centering
				\includegraphics[width=\linewidth]{Recursos/NuevoEsquemaParticiones.png}
				\captionof{figure}{Nuevo esquema de particiones}
				\label{NuevasParticiones}
			\end{minipage}
			\hfill
			\begin{minipage}{0.55\textwidth}
				\centering
				\includegraphics[width=\linewidth]{Recursos/ErrorDisco.png}
				\captionof{figure}{Error 0x80300001 = “Windows no ve el disco que acabas de seleccionar”, reconoce el disco porque está cargado en RAM pero realmente no puedo isntalar el \textit{WS}, por lo que toca volver a ir al apartado de hardware y seleccionar la ISO de \textit{Windows Server} para que ya reconozca el disco y se pueda proceder con la instalación}
				\label{ErrorInstalacion}
			\end{minipage}
		\end{center}
		\begin{figure}[H]
			\centering
			\includegraphics[width=0.6\linewidth]{Recursos/InstalacionCompletada.png}
			\caption{En este punto he tenido que establecer una contraseña para poder acceder al usuario adminitrador en el \textit{WS} }
		\end{figure}
	\subsection{Intalación de drivers}
		\begin{center}
			\begin{minipage}{0.55\textwidth}
				\centering
				\includegraphics[width=\linewidth]{Recursos/driversv2}
				\captionof{figure}{Seleccionaremos los drivers de red.}
				\label{seleccionDriversRed}
			\end{minipage}
			\hfill
			\begin{minipage}{0.4\textwidth}
				Una vez que ya hemos finalizado la instalación, me he dado cuenta de que faltan funcionalidades como puede ser la de red.\\ Al igual que hicimos para que el instalador reconociese los discos, volvemos a cargar \textit{Virtio.ISO} en el apartado de \textit{hardware}. Posteriormente vamos al \textbf{Administrador de dispositivos} y actualizamos los controladores de de todos aquellos dispositivos que aparezcan con una exclamación (drivers de red, dispositivos de comunicación simples PCI y dispositivo PCI). El proceso es el mismo 
			\end{minipage}
		\end{center}
		Los dispositiovs que nos fallaban se encargan de: 
		\begin{itemize}
			\item \textit{Controladora Ethernet (Ethernet Controller)}: Necesario para la comunicación con la red.
			\item \textit{Controladora simple de comunicaciones PCI (Simple Communications Controller PCI)}: Relacionado con las tareas de mantenimiento y de administración remota.
			\item \textit{Dispositivo PCI (PCI Device)}: Dedicado a la comunicación entre el dispositivo conectado por PCI (una tarjeta de sonido por ejemplo).
		\end{itemize}
		
	\subsection{Administrador del servidor}
		Una vez que tenemos todo instalado (de momento), vamos a investigar qué es el panel de \textit{Administrador del servidor.}
		\subsection{Servidor Local}
		\begin{figure}[H]
			\centering
			\includegraphics[width=0.6\linewidth]{Recursos/ServerLocal}
			\caption{En este menú vemos toda la información relacionada con nuestra máquina virtual.}
			\label{ServerLocal}
		\end{figure}
		En esta imagen podemos mirar que tenemos un procesor típico de servidores, un firewall activado y la función de escritorio remoto la tenemos desactivada. 
			\begin{center}
				\begin{minipage}{0.55\textwidth}
					\centering
					\includegraphics[width=\linewidth]{Recursos/enableRDP}
					\captionof{figure}{Habilitamos el servicio de escritorio remoto}
					\label{enableRDP}
				\end{minipage}
				\hfill
				\begin{minipage}{0.4\textwidth}
					Esta es la versión propietaria de \textit{Microsoft} llamada \textit{\textbf{RDP} (Remote Desktop Protocol)} de tal manera que permite a nuestra máquina establecer una conexión remota con otro pc que se encuentre dentro de una red, permitiéndote el acceso total al mismo, lógicamente esto está desactivado por motivos de seguridad, pero como nosotros vamos a administrar ese servicio \textbf{procedemos a habilitar el escritorio remoto.} con el consecuente aviso de que se va a abrir una excepción en el firewall para que se pueda emplear este servicio.
				\end{minipage}
			\end{center}
		\subsubsection{Cambio del nombre del equipo y del grupo de trabajo}
			\begin{center}
				\begin{minipage}{0.55\textwidth}
					\centering
					\includegraphics[width=\linewidth]{Recursos/CambioNombreYWorkGroup}
					\captionof{figure}{Cambiamos el nombre de la máquina y el del grupo de trabajo}
					\label{WorkGroup}
				\end{minipage}
				\hfill
				\begin{minipage}{0.4\textwidth}
					¿Qué significa que hagamos estos cambios?\\ 
					El nombre del Equipo es como se nos va a poder identificar dentro de la red, mientras que el  \textit{miembro de} indica a qué grupo de trabajo va a pertenecer nuestro servidor.\\ En mi caso, como no existe ninguno, lo que hace el OS internamente es crear un grupo de trabajo vacío y se une a él.
				\end{minipage}
			\end{center}
		\subsubsection{Información sobre la red}
			\begin{center}
			\begin{minipage}{0.55\textwidth}
				\centering
				\includegraphics[width=\linewidth]{Recursos/NetWorkInfo}
				\captionof{figure}{Información de la dirección IPv4 de nuestra máquina}
				\label{IPv4}
			\end{minipage}
			\hfill
			\begin{minipage}{0.4\textwidth}
				Siguiendo el guión tenemos que mirar la información de la dirección IP de nuestra máquina, si nos fijamos en la captura, sólo tenemos configurada un dirección IPv4, la cual es la \textbf{10.0.38.2}, pero ojo, esta es una dirección IP \underline{privada} es decir, identifica a la máquina virtual pero dentro de la subred (LAN) de la UVa(más probablemente de los servidores de la escuela de ingeniería informática), no de cara a internet, de tal manera que para poner conectarnos a internet con esta IP el cortafuegos deberá traducir esta IPprivada a una pública que sea accesible desde Internet, para lo cuál utiliza la máscara de subred, para identificar la red y el dispositivo dentro de la misma.
			\end{minipage}
		\end{center}
		\subsubsection{Cambio de puerto del protocolo \textit{RDP}}
			\begin{center}
			\begin{minipage}{0.55\textwidth}
				\centering
				\includegraphics[width=\linewidth]{Recursos/WKPRDP}
				\captionof{figure}{Cambio de puerto del \textit{RDP}}
				\label{WKPRDP}
			\end{minipage}
			\hfill
			\begin{minipage}{0.4\textwidth}
				Como se nos indica en el guión, el \textit{RDP}, emplea \textit{tcp} el puerto 3389, lo cual \underline{\textit{WKP}}  \textit{Well Know Port} (puertos cuyo uso está estandarizado para ciertos servicios) puesto que se sale del rango de los 1025 (0-1024) primeros puertos,  pero sí es conocido en los sistemas de \textit{Microsoft}.\\
			El que no sea uno de los \textit{WKP} es un arma de doble filo, porque permite a un usuario sin permisos de administrador (el equivalente a \textit{sudo} en el OS de Linux)  conectarse de manera remota a nuestra máquina, lo que hace que si se sufre un ataque nuestro \textit{host} que expuesto de manera sencilla.\\  Lo que sí que es una ventaja es que el OS no tiene que emplear una multiplexión de puertos para redirigir las peticiones en caso de que otro servicio requiera el uso de dicho puerto.\\ 
			En mi caso, prefiero mover el servicio a un \textit{WKP} para forzar que el usuario tenga permisos de administrador o de \textit{sudo} para realizar la conexión remota
			\end{minipage}
		\end{center}
	\subsection{Configuración del firewall}
		Si recordamos, en la el servidor local (\ref{ServerLocal}) tenemos habilitado un firewall, vamos a proceder a configurarlo, para permitir conexiones.
			\begin{figure}[H]
				\centering
				\includegraphics[width=0.6\linewidth]{Recursos/Firewall1}
				\caption{Entramos en la opción de firewall y nos vamos a opciones avanzadas}
				\label{FirewallBase}
			\end{figure}
		\subsubsection{Puerto 22$\rightarrow$ tcp/22}
			Este puerto es el estándar para permitir conexiones mediante el protocolo \textit{SSH (Secure Shell)}, el cual permite establecer conexiones remotas entre \textit{hosts}, traducido sería decir que nos permite conectarnos de manera remota a otro dispositivo.\\ Por defecto el Firewall no permite conexiones a ese puerto.
			\begin{center}
				\begin{minipage}{0.45\textwidth}
					\centering
					\includegraphics[width=\linewidth]{Recursos/regla221}
					\captionof{figure}{Primer paso para establecer la regla, decir que vamos a aplicarla sobre un puerto}
				\end{minipage}
				\hfill
				\begin{minipage}{0.45\textwidth}
					\centering
					\includegraphics[width=\linewidth]{Recursos/regla222}
						\captionof{figure}{Especificamos el protocolo y el puerto, en nuestro caso tcp (puesto que a ser seguro necesita establecer el \textit{handshake} y confirmación de recepción) y el puerto 22}
				\end{minipage}
			\end{center}
			\begin{center}
				\begin{minipage}{0.45\textwidth}
				\centering
				\includegraphics[width=\linewidth]{Recursos/regla223}
					\captionof{figure}{Establecemos el dominio de la norma, para la máquina virtual, en cualquier caso }
				\end{minipage}
				\hfill
				\begin{minipage}{0.45\textwidth}
					\centering
					\includegraphics[width=\linewidth]{Recursos/regla224}
						\captionof{figure}{Le añadimos un nombre y descripción para poder localizarla una vez que se ha añadido}
				\end{minipage}
			\end{center}
			\subsection{Prueba de conexión remota empleando la aplicación Escritorio Remoto}
	\section{Configuración del \textit{HIPERVISOR HYPER-V} }
		\subsection{¿Qué son los Roles en un \textit{Windows Server}}
			En el contexto de un servidore son \textbf{el conjunto primario de servicios de software y configuraciones}, un ejemplo de esto es el \textit{Active Directory} o un \textit{servidor DHCP}. De tal manera que hace que el servidor se encargue de una tarea específica.
		\subsection{¿Qué son las caracterísiticas en \textit{Windows Server}}
			Son programas o herramientas que  o bien complementan las funciones de un \textit{Role} o que proporcionan herramientas las cuales son útiles de manera independiente. En el contexto de un \textit{WS} tenemos como ejemplo el \textit{.NET Framework} que es esencial para algunas herramientas de diagnóstico o gestión.\\ Ojo, que estas características son para los administradores del sistema no tienen por qué ser para los usuarios del mismo.
		\subsubsection{Agregar \textit{roles} y \textit{características} a nuestro servidor}
			\begin{figure}[H]
				\centering
				\begin{minipage}{0.45\textwidth}
					\centering
					\includegraphics[width=\textwidth]{Recursos/AgregarRol}
					\caption{Menú para seleccionar los \textit{roles} y \textit{características}}
					\label{annadirroles}
				\end{minipage}
				\hfill
				\begin{minipage}{0.45\textwidth}
					\centering
					\includegraphics[width=\textwidth]{Recursos/hypervisor}
					\caption{Instalamos las caracteristicas que se nos indica en el guión tras seleccionar las intalaciones por roles o características}
					\label{instalarRecursos}
				\end{minipage}
			\end{figure}
			\begin{center}
				\begin{minipage}{0.4\textwidth}
					\centering
					\includegraphics[width=\linewidth]{Recursos/Conmutador}
					\captionof{figure}{Seleccionaremos el conmutador como indica el guión.}
					\label{conmutadohyper}
				\end{minipage}
				\hfill
				\begin{minipage}{0.55\textwidth}
					\textit{Hyper-V} es una tecnología de virtualización de Windows, que permite alojar y ejecutar otras \textit{Virtual Machines}(VM).\\
					El \textbf{conmutador de red}  funciona como un \textit{switch} permitiendo que las máquinas virtuales se comuniquen entre sí y que las mismas tengan conexión a internet (empleando nuestra tarjeta de red para ello).
				\end{minipage}
			\end{center}
			\begin{figure}[H]
				\centering
				\includegraphics[width=0.6\linewidth]{Recursos/ConfirmarInstalacion}
				\caption{Un resumen de lo que vamos a instalar en nuestro WS, tras la instalación toca reiniciar la máquina para que se apliquen estos cambios.}
			\end{figure}
		\subsubsection{Importar \textit{VM} disponibles}
			Tras permitir la \textit{detección de redes} (es simplemente abrir el menú y darle a la opción) permitiendo ser detectado por otros dispositivos de la red y poder acceder a tus recursos compartidos, aunque esto sea un submenú dentro de la carpeta de red, en verdad lo que estamos haciendo es configurar el \textit{firewall} y a la vez expone a nuestra máquina a ser descubierta por una exploración básica de la red.
			\begin{figure}[H]
				\centering
				\begin{minipage}{0.45\textwidth}
					\centering
					\includegraphics[width=\textwidth]{Recursos/accederACarpetasCompartidas}
					\caption{Introducimos las credenciales de acceso para acceder a esta máquina Usuario: usuario y contraseña: Passw0rd}
					\label{accederCarpetasCompartidas}
				\end{minipage}
				\hfill
				\begin{minipage}{0.45\textwidth}
					\centering
					\includegraphics[width=\textwidth]{Recursos/ContenidoVM}
					\caption{Los directorios para las máquinas virtuales, como se indica en el guión copiamos \textit{TinyWin11} a nuestro equipo}
					\label{vmdisponibles}
				\end{minipage}
			\end{figure}
			\begin{figure}[H]
				\centering
				\begin{minipage}{0.45\textwidth}
					\centering
					\includegraphics[width=\textwidth]{Recursos/importar1}
					\caption{Vamos al menú de administrar el \textit{Hyper-V} click derecho sobre nuestro grupo e importar máquina virtual}
					\label{importacion1}
				\end{minipage}
				\hfill
				\begin{minipage}{0.45\textwidth}
					\centering
					\includegraphics[width=\textwidth]{Recursos/importar2}
					\caption{Las opciones que nos da para importar la \textit{VM}}
					\label{importacion2}
				\end{minipage}
			\end{figure}
			Si nos damos cuenta, aparecen varias opciones para importar la máquina virtual, voy a explicarlas poco a poco:
			\begin{itemize}
				\item Registrar la máqina virtual (usar el identificador único existente): Reutiliza la \textit{VM} exactamente como fue exportada, es decir, la copiamos tal cual.
				\item Restaurar la máquina virtual (usar el identificador único existente): Reemplaza una \textit{VM} existente por una nueva versión, sobreescribiendo la máquina antigua si es que comparten el mismo \textit{ID}
				\item Copiar la máquina virtual (crear un identificador único nuevo): Crea una nueva copia de la \textit{VM} (creando un nuevo identificador único), de tal manaera que la clonamos para usarla como dos instancias diferentes.
			\end{itemize}
			\begin{figure}[H]
				\centering
				\begin{minipage}{0.45\textwidth}
					\centering
					\includegraphics[width=\textwidth]{Recursos/importar3}
					\caption{Se ha finalizado la instalación}
					\label{importacion3}
				\end{minipage}
				\hfill
				\begin{minipage}{0.45\textwidth}
					\centering
					\includegraphics[width=\textwidth]{Recursos/TiniWindows11}
					\caption{Damos a conectar la máquina virtual desde el menú de nuestra máquina en el administrador de \textit{hyper-v}}
					\label{funcionavmWindows}
				\end{minipage}
			\end{figure}
		\subsubsection{Instalación de la VM con Ubuntu}
		Vamos a repetir el proceso, copiamos la imagen desde la máquina de red y volvemos al menú de \textit{hyper-v}
		\begin{figure}[H]
			\centering
			\begin{minipage}{0.45\textwidth}
				\centering
				\includegraphics[width=\textwidth]{Recursos/importar6}
				\caption{Comenzamos el proceso de importar la máquina virtual de Ubuntu}
				\label{importacion6}
			\end{minipage}
			\hfill
			\begin{minipage}{0.45\textwidth}
				\centering
				\includegraphics[width=\textwidth]{Recursos/importar5}
				\caption{Damos a conectar la máquina virtual desde el menú de nuestra máquina en el administrador de \textit{hyper-v}}
				\label{funcionavmUbutnu}
			\end{minipage}
		\end{figure}
		
		\begin{figure}[H]
			\centering
			\includegraphics[width=0.6\textwidth]{Recursos/AmbasVMCorriendo}
			\caption{Ambas \textit{VM} corriendo en el el \textit{WS}}
		\end{figure}
	\subsection{Ips de la máquinas virtuales en el servidor}
	\begin{figure}[H]
		\centering
		\begin{minipage}{0.45\textwidth}
			\centering
			\includegraphics[width=\textwidth]{Recursos/DireccionIPUBUNTU}
			\caption{Ejecución del comando \textit{ip addr} para sacar la dirección ip de la \textit{VM}}
			\label{UbuntuIP}
		\end{minipage}
		\hfill
		\begin{minipage}{0.45\textwidth}
			\centering
			\includegraphics[width=0.6\textwidth]{Recursos/DireccionIPWindows}
			\caption{Dirección IP de la máquina con Windows}
			\label{IpWindows}
		\end{minipage}
	\end{figure}
	\subsection{Comunicación entre \textit{VM}s dentro del servidor}
		\subsubsection{Configuración previa en la máquina virtual Windows 11}
			\begin{figure}[H]
			\centering
			\begin{minipage}{0.45\textwidth}
				\centering
				\includegraphics[width=\textwidth]{Recursos/openssh}
				\caption{Intalacion del servicio openssh para windows }
			\end{minipage}
			\hfill
			\begin{minipage}{0.45\textwidth}
				\centering
				\includegraphics[width=\textwidth]{Recursos/opensshservice}
				\caption{Establecemos que el servicio se inicie de manera automática con la máquina.}
			\end{minipage}
		\end{figure}
		\begin{figure}[H]
			\centering
			\includegraphics[width=\textwidth]{Recursos/rule}
			\caption{Aplicamos esta regla en el firewall para que permita el trafíco por el puerto 22.}
		\end{figure}
		\subsubsection{\textit{sshs} entre vms}
		\begin{figure}[H]
			\centering
			\begin{minipage}{0.45\textwidth}
				\centering
				\includegraphics[width=\textwidth]{Recursos/pingWindowsUbuntu}
				\caption{ssh de la máquina Ubuntu a la de Windows11}
				\label{Ubutnu}
			\end{minipage}
			\hfill
			\begin{minipage}{0.45\textwidth}
				\centering
				\includegraphics[width=\textwidth]{Recursos/pingWindowsUbuntu}
				\caption{ssh de la máquina Windows 11 a la Ubuntu}
				\label{conectar}
			\end{minipage}
		\end{figure}
	No he conseguido hacer un ping de la máquina ubuntu a la windows, por ninguno de los puertos y no me devuelve conectividad 
	\subsection{Comunicacion desde el servidor a las dos \textit{VM}s}
	En la máquina con ubuntu es tan sencillo como hacer un ssh desde el \textit{PowerShell} con permisos de administrador, sin embargo como no he conseguido hacer funcionar el servicio \textit{OpenSSH}  he  empleado la conexión al escritorio remoto de windows (habilitando previamente esta característica en la máquina virtual.)
	\begin{figure}[H]
		\centering
		\begin{minipage}{0.45\textwidth}
			\centering
			\includegraphics[width=\textwidth]{Recursos/sshServerUbuntu}
			\caption{ssh desde el \textit{WS} a la \textit{VM} con Ubuntu}
			\label{UbutnuDesdeServer}
		\end{minipage}
		\hfill
		\begin{minipage}{0.45\textwidth}
			\centering
			\includegraphics[width=\textwidth]{Recursos/EscritorioRemotoExitoso}
			\caption{Conexión del escritorio remoto desde nuestro \textit{WS} a la máquina virtual de Windows }
			\label{conectarWindowsAWindows}
		\end{minipage}
	\end{figure}
	\subsection{Creación de una red interna para las máquinas virtuales anidades}
		\subsubsection{Creación de un conmutador  virtuales}
			\begin{figure}[H]
				\centering
				\begin{minipage}{0.45\textwidth}
					\centering
					\includegraphics[width=\textwidth]{Recursos/CreacionConmutadorVirtual}
					\caption{Tenemos que entender que los conmutadores, son un equivalente a las interfaces de red}
					\label{CreacionConmutador}
				\end{minipage}
				\hfill
				\begin{minipage}{0.45\textwidth}
					\centering
					\includegraphics[width=\textwidth]{Recursos/ConmutadorCreado}
					\caption{Creamos un conmutador que sólo va a funcionar como una red interna para nuestras \textit{VM}s }
					\label{ConmutadorPrivado}
				\end{minipage}
			\end{figure}
			
		\subsubsection{Propiedades de la red}
			\begin{figure}[H]
				\centering
				\begin{minipage}{0.45\textwidth}
					\centering
					\includegraphics[width=\textwidth]{Recursos/RedPrivada}
					\caption{La direcciónIP del conmutador que hemos creado}
					\label{IPConmutador}
				\end{minipage}
				\hfill
				\begin{minipage}{0.45\textwidth}
					\centering
					\includegraphics[width=\textwidth]{Recursos/AdaptadorCreadoConExito}
					\caption{Si nos vamos al apartado de recursos compartidos, vemos que se ha creado una nueva conexión de red.}
					\label{NuevaTarjetaRed}
				\end{minipage}
			\end{figure}
			
		\subsubsection{Cambio del adaptador de red}
			\begin{figure}[H]
				\centering
				\includegraphics[width=0.6\textwidth]{Recursos/CambioAdaptador}
				\caption{Establecemos el nuevo adaptador como el predefinido para las dos máquinas virutales}
			\end{figure}
		\subsubsection{Configuración del adaptador de red en las \textit{VM}s}
			\begin{figure}[H]
				\centering
				\begin{minipage}{0.45\textwidth}
					\centering
					\includegraphics[width=\textwidth]{Recursos/redWindowsInterna}
					\caption{Configuramos la ip estática de la máquina a mano, para que ya no consiga la ip del servidor DHCP sino de lo que le acabamos de escribir}
					\label{IPEstaticaWindows}
				\end{minipage}
				\hfill
				\begin{minipage}{0.45\textwidth}
					\centering
					\includegraphics[width=\textwidth]{Recursos/UbuntuConfiguracion}
					\caption{editamos el fichero para establecer los mismos ajustes a mano que en el windows, tras esto debemos aplicar el comando \textit{netplay apply}}
					\label{IPestaticaUbuntu}
				\end{minipage}
			\end{figure}
		\subsection{Problema con el \textit{netplay apply}}
				\begin{center}
				\begin{minipage}{0.4\textwidth}
					\centering
					\includegraphics[width=\linewidth]{Recursos/Erroryml}
					\captionof{figure}{Al ejecutar el comando me sale ese error.}
					\label{applyError}
				\end{minipage}
				\hfill
				\begin{minipage}{0.55\textwidth}
					Si miro el formato del fichero que estoy editando, me doy cuenta de que es un \textit{.yaml} el cuál es un lenguaje muy, pero muy, estricto con la identiación y el formato, el código del fichero, es correcto, lo que falta es identarlo de manera correcta, cada nivel tiene que tener exactamente 2 espacio hacía el interior y después de los : tiene que haber un espacio, así como tiene que haber un esapcio entre el - y el to.
				\end{minipage}
			\end{center}
		\subsubsection{Solución del problema en el fichero \textit{yaml}}
			\begin{figure}[H]
			\centering
			\begin{minipage}{0.45\textwidth}
				\centering
				\includegraphics[width=\textwidth]{Recursos/ymkcorrecto}
				\caption{Corrijo los errores de indentación y sintaxis y ejecuto el comando para que se pueda aplicar el cambio.}
				\label{yamlCorregido}
			\end{minipage}
			\hfill
			\begin{minipage}{0.45\textwidth}
				\centering
				\includegraphics[width=\textwidth]{Recursos/Funciona}
				\caption{Ejecuto el comando \textit{ip addr show eth0} para ver que ahora la ip de la máquina Ubuntu es la que deseo tener de manera estática.}
				\label{IPestativaUbuntu}
			\end{minipage}
		\end{figure}
		\subsection{Conexiones entre las máquinas con el cambio de ip}
			\begin{figure}[H]
				\centering
				\includegraphics[width=0.6\textwidth]{Recursos/establacerADaptadorInterna}
				\caption{Creamos la regla que hace que todo lo que vaya hacía al red interna de nuestras máquinas virutales pase por la interfaz que hemos creado en el \textit{Hyper-V}}
			\end{figure}
			Adicionalemnte me doy cuenta de que el adaptador no tiene una red definida, por lo que la defino con los siguientes comando en \textit{PowerShell} como administrador
			\begin{center}
				\textit{New-NetIPAddress -InterfaceAlias "vEthernet (RedInterna)" -IPAddress 192.168.2.1 -PrefixLength 24}
				\textit{Set-NetConnectionProfile -InterfaceAlias "vEthernet (RedInterna)" -NetworkCategory Private}
			\end{center}
			\subsubsection{Comunicación entre el \textit{WS} y las \textit{VM}}
				\begin{figure}[H]
					\centering
					\begin{minipage}{0.45\textwidth}
						\centering
						\includegraphics[width=\textwidth]{Recursos/sshWSw11}
						\caption{Establezco el \textit{RDP} en la dirección 192.168.2.100}
						\label{VMWindows}
					\end{minipage}
					\hfill
					\begin{minipage}{0.45\textwidth}
						\centering
						\includegraphics[width=\textwidth]{Recursos/sshWSUbuntuIPEstatia}
						\caption{Ejecuto el comando \textit{ip addr show eth0} para ver que ahora la ip de la máquina Ubuntu es la que deseo tener de manera estática la cual es 192.168.2.101.}
						\label{VMUbuntu}
					\end{minipage}
				\end{figure}
			\subsubsection{Conexión de la máquina Ubuntu a la windows}
				\begin{figure}[H]
						\centering
						\includegraphics[width=0.6\textwidth]{Recursos/sshWindowsUbuntu}
						\caption{Conexión ssh desde la máquina windows a la máquina con Ubuntu}
				\end{figure}
\section{Instalar el servicio remoto \textit{RAS}}
\begin{center}
	\begin{minipage}{0.4\textwidth}
		\centering
		\includegraphics[width=\linewidth]{Recursos/InstalacionRAS}
		\captionof{figure}{Instlación de los servicios RAS}
		\label{RAS}
	\end{minipage}
	\hfill
	\begin{minipage}{0.55\textwidth}
		\textit{RAS} son las siglas de \textit{Remote Access Service} el cuál nos permite acceder de manera remota y segura a redes corporativas, es decir hace las funciones de una \textbf{VPN} o un \textit{DA} (\textit{Direct Access}) el cual es una manera de conectarse directamente a al red coorporativa sin necesidad de instalarse una VPN.\\
		Además la característica de enrutamiento lo que hace es permitir la comunicación de paquetes entre diferentes fragmentos de la red.\\ De tal manera que vamos a tener una \textit{gateway} para poder comunicarnos con los servidores de una red coorporativa sin tener que estar conectados a la red física (\textit{ethernet}) de la misma.\\
		Tras la instalación reiniciamos el servidor para poder empezar a comunicarnos.
	\end{minipage}
\end{center}
	\subsection{Configuración del \textit{RAS}}
		\begin{figure}[H]
			\centering
			\includegraphics[width=0.6\linewidth]{Recursos/RAS}
			\caption{Vamos a la aplicación de \textbf{Enrutamiento y acceso remoto} y configuramos el DC}
		\end{figure}
			\begin{figure}[H]
			\centering
			\begin{minipage}{0.45\textwidth}
				\centering
				\includegraphics[width=\textwidth]{Recursos/TraduccionNAT}
				\caption{Permite que los host de nuestro servidor, las máquinas virtuales, se conecten a internet empleando IPs públicas}
				\label{TraduccionNAT}
			\end{minipage}
			\hfill
			\begin{minipage}{0.45\textwidth}
				\centering
				\includegraphics[width=\textwidth]{Recursos/TraduccionNAT2}
				\caption{Permitimos emplear las interfaces de redes que ya tenemos para conectarse a internet.}
				\label{TraduccionNAT1}
			\end{minipage}
		\end{figure}
			\begin{figure}[H]
			\centering
			\begin{minipage}{0.45\textwidth}
				\centering
				\includegraphics[width=\textwidth]{Recursos/TraduuccionNAT3}
				\caption{	En este menú estamos seleccionando cómo queremos obtener los servicios \textit{DHCP} (Cómo obtenemos las direcciones IP públicas de nuestra red) y \textit{DNS} (el sistema que traduce de urls a direcciones IP).}
				\label{TraduccionNAT3}
			\end{minipage}
			\hfill
			\begin{minipage}{0.45\textwidth}
				\centering
				\includegraphics[width=\textwidth]{Recursos/AsignacionDireccionesNAT}
				\caption{El rango de direcciones que tenemos para nuestas conexiones remotas}
				\label{direcciones}
			\end{minipage}
		\end{figure}
		\subsubsection{Prueba de \textit{curl}}
			\begin{center}
				\begin{minipage}{0.4\textwidth}
					\centering
					\includegraphics[width=\linewidth]{Recursos/curl}
					\captionof{figure}{Hacemos una petición curl a una dirección ip porque aún no tenemos configurado el DNS en nuestra máquina}
					\label{curl}
				\end{minipage}
				\hfill
				\begin{minipage}{0.55\textwidth}
					\textit{curl} es una herramienta que nos permite lanzar peticiones \textit{http}/s y podemos cargar y descargar archivos desde servidores remotos, en el caso de la captura he descargado el código html de la página web a la que he lanzado la petición, aunque otro uso es probar APIs puesto que es una manera liviana y simple para que por línea de comandos podamos ver cómo y qué responde un servidor a una petición \textit{http/s}
				\end{minipage}
			\end{center}
\section{Instalar el servicio DNS en nuestro \textit{WS2022}}
		\begin{center}
		\begin{minipage}{0.4\textwidth}
			\centering
			\includegraphics[width=\linewidth]{Recursos/InstalarServicioNombresAC}
			\captionof{figure}{Proceso habitual desde el \textit{Hyper-V} descargamos el servicio de nombres para el \textit{AC}}
			\label{SNAC}
		\end{minipage}
		\hfill
		\begin{minipage}{0.55\textwidth}
			Me pregunto ¿por qué es necesario descargarse otro servicio más? la respuesta a esta pregunta, es porque el \textit{ADDS} (\textit{Active Directory Domain Services}) es clave para poder tener una gestión centralizada y jerárquica de todos los recursos, ya sean hardware o de usuarios. \\ Centralizando la autenticación y con esa cuenta el usuario puede acceder a cualquier servicio dentro de nuestro servidor. Además de instalar el \textit{DNS} para poder encontrar los recursos dentro de nuestra red sin tener que aprenderse las direcciones IPs.
		\end{minipage}
	\end{center}
	\begin{figure}[H]
		\centering
		\begin{minipage}{0.45\textwidth}
			\centering
			\includegraphics[width=\textwidth]{Recursos/Promote}
			\caption{Promover implica convertir el \textit{WS2022} en el centro de control de la red. Es decir va a asumir toda la autoridad sobre nuestra red.}
			\label{VMWindowsPromote}
		\end{minipage}
		\hfill
		\begin{minipage}{0.45\textwidth}
			\centering
			\includegraphics[width=\textwidth]{Recursos/arbol}
			\caption{Como estamos haciendo un \textit{AC} desde cero, la opción que no interesa es la seleccionada, puesto que la primera es para añadir servidores y la segunda es para añadir dominios hijos a los que ya tenemos.}
			\label{arbol}
		\end{minipage}
	\end{figure}
	\begin{figure}[H]
		\centering
		\begin{minipage}{0.45\textwidth}
			\centering
			\includegraphics[width=\textwidth]{Recursos/nuevoArbol}
			\caption{Estamos definiendo la versión mínima de Windows Server que soportará la estructura de \textit{AC}, lo cual determina las características de \textit{AD DS} que estarán disponibles. \textbf{Contraseña: ASO\_2025WIndowsServer}.}
			\label{margarita}
		\end{minipage}
		\hfill
		\begin{minipage}{0.45\textwidth}
			\centering
			\includegraphics[width=\textwidth]{Recursos/NetBios}
			\caption{\textit{Network Basic Input/Output System} es un sistema para mejorar la compatiblidad consistemas antiguos, además de que sirve como un alias corto dentro de la red. Tras cofigurar esto, le damos a siguiente, siguiente hasta que nos aparezca la opción de instalar, posteriormente reiniciamos la máquina.}
			\label{NetBiosl}
		\end{minipage}
	\end{figure}

		\subsubsection{¿Qué es un nivel funcional en el contexto del \textit{Active Directory}?}
			 Es la versión mínima de OS para que pueda emplear los servicios del servidor. Es decir, si nos fijamos en \ref{arbol}, la versión mímima del OS para que el server pueda funcionar como \textit{DC} (Controlador de Dominio) de tal manera que si alguna máquina se intenta unir a nuestra red o ser un árbol de nuestro directorio, tendrá que ser una versión posterior a la del 2016 (está inclusive), con esto, nos aseguramos de poder ofrecer servicios más modernos y reducimos el riesgo de tener vulnerabilidades, puesto que cuanto más modernos sea la versión del OS menos vulnerabilidades documentadas habrá en la red para realizar ataques, así como menos tiempo tendrá un atacante para concer y explotar vulnerabilidades de las diferentes \textit{releases}
	 
	 \subsection{Revisión de si el cambio de nuestro servidor al rol de \textit{AD} ha sido realizada correctamente}
		 \begin{figure}[H]
		 		\centering
		 		\includegraphics[width=0.6\linewidth]{Recursos/PromocionCorrecta}
		 		\captionof{figure}{Pantallazo de las propiedades del servidor local como controlador de \textit{AD}}
		 		\label{ADServerl}
		 \end{figure}
		 \begin{center}
		 	\begin{minipage}{0.4\textwidth}
		 		\centering
		 		\includegraphics[width=\linewidth]{Recursos/AppDNS}
		 		\captionof{figure}{Servicio DNS en el \textit{WS22}}
		 		\label{DNS}
		 	\end{minipage}
		 	\hfill
		 	\begin{minipage}{0.55\textwidth}
		 		\textit{ad-aso02.lab} es la zona de traducción de direcciones ips a nombres de dominio. Además de indicar que nuestro servidor va a ser la fuente de información para el DNS.\\
		 		Mientras que las carpetas que se ven en el directorio son las necesarias para que los clientes u otros serivicios puedan encontrar los servicios clave del \textit{AD}.\\ 
		 		Este servicio es crítico puesto que sin el DNS, ningún servicio/pc/cliente sabría como comunicarse con los recursos que necesita.
		 	\end{minipage}
		 \end{center}
	\subsection{Creación de una nueva zona DNS para la búsqueda inversa}	
		 \begin{center}
			\begin{minipage}{0.4\textwidth}
				\centering
				\includegraphics[width=\linewidth]{Recursos/DNSInverso1}
				\captionof{figure}{Servicio DNS inverso en el \textit{WS22}}
				\label{DNSInverso}
			\end{minipage}
			\hfill
			\begin{minipage}{0.55\textwidth}
				El DNS Inverso es el proceso de traducción de una dirección IP a nombre del dominio, es decir traducir la 192.168.2.100 a \textit{TiniWindows}, esto ofrece ventajas de seguridad, pues no permite saturar la red con direcciones ips que no tienen asignadas un nombre de dominio  y además nos permite tener un monitoreo de red más legibles puesto que ya no tenemos que memorizas direcciones ips sino nombres de dominio lo que resulta más sencillo puesto que están escritas en lenguaje humano y como una una ristra de números.
			\end{minipage}
		\end{center}
		\begin{figure}[H]
			\centering
			\begin{minipage}{0.45\textwidth}
				\centering
				\includegraphics[width=\textwidth]{Recursos/DNSInverso2}
				\caption{Identificamos el rango de direcciones IPs sobre la que vamos a hacer la traducción inversa, en mi caso \textit{192.168.2.xxx} para abarcar toda la subred.}
				\label{NuevaZona}
			\end{minipage}
			\hfill
			\begin{minipage}{0.45\textwidth}
				\centering
				\includegraphics[width=\textwidth]{Recursos/DNSInverso3}
				\caption{Hemos creado la zona, y esta se propagará a todos los \textit{DC} }
				\label{ZonaCreada}
			\end{minipage}
		\end{figure}
		\begin{center}
			\begin{minipage}{0.4\textwidth}
				\centering
				\includegraphics[width=\linewidth]{Recursos/Puntero}
				\captionof{figure}{Servicio DNS en el \textit{WS22}}
				\label{Puntero}
			\end{minipage}
			\hfill
			\begin{minipage}{0.55\textwidth}
				Un puntero es la creación de una zona en específico dentro de la zona de búsqueda inversa que hemos creado. Mapean la dirección ip a un nombre de dominio específico. las dosque hemos creado son para la gateway y para la \textit{VM} de windws.
			\end{minipage}
		\end{center}
	\subsubsection{Añade en la zona directa los registros de tipo A}
		Los registros de tipo A son aquellos que añaden un host y su dirección ip dentro de nuestro servicio DNS, identíficándolo de manera única y ofreciendo a los clientes una manera directa de acceder a ellos.
		\begin{figure}[H]
			\centering
			\begin{minipage}{0.45\textwidth}
				\centering
				\includegraphics[width=\textwidth]{Recursos/hostAUbuntu}
				\caption{HostA para poder acceder a la máquina Ubuntu con IP \textit{192.168.2.101}}
				\label{AUbuntu}
			\end{minipage}
			\hfill
			\begin{minipage}{0.45\textwidth}
				\centering
				\includegraphics[width=\textwidth]{Recursos/hostAWindows}
				\caption{ HostA para poder acceder a la máquina windows con la dirección IP \textit{192.168.2.100}}
				\label{AWindows}
			\end{minipage}
		\end{figure}
	\subsubsection{Agregar \textit{TinyWindows} a nuestro dominio}
		\begin{figure}[H]
			\centering
			\begin{minipage}{0.45\textwidth}
				\centering
				\includegraphics[width=\textwidth]{Recursos/TinyWindows}
				\caption{Añadimos la máquina virtual \textit{TinyWindows} como un miembro del dominio}
				\label{twdominio}
			\end{minipage}
			\hfill
			\begin{minipage}{0.45\textwidth}
				\centering
				\includegraphics[width=\textwidth]{Recursos/Exito}
				\caption{Tras darle a aceptar aparecerá un popup que nos pide un usuario y contraseña con permisos en el dominio, tecleamos la de la cuenta de administrador del \textit{WS} y ya hemos terminado}
				\label{twexito}
			\end{minipage}
		\end{figure}
	\subsubsection{Agregar la \textit{VM} con Ubuntu al dominio}
	\begin{center}
		\begin{minipage}{0.4\textwidth}
			\centering
			\includegraphics[width=\textwidth]{Recursos/instalacionCosasUbuntu}
			\captionof{figure}{Tras actualizar el sistema, vemos que tenemos conexión a internet y me instalo todos los paquetes que menciona el guión.}
			\label{fig:instalacion}
		\end{minipage}
		\hfill
		\begin{minipage}{0.55\textwidth}
			\begin{itemize}
				\item \textbf{\texttt{sssd}}: Demonio central (\textit{System Security Services Daemon}) para la autenticación e identificación remota de usuarios y grupos (AD, LDAP), con soporte de caché.
				\item \textbf{\texttt{sssd-tools}}: Contiene las herramientas de línea de comandos necesarias para la administración y el diagnóstico del servicio SSSD (por ejemplo, \texttt{sssctl}).
				\item \textbf{\texttt{libnss-sss}}: Módulo del \textit{Name Service Switch} (NSS). Permite al sistema operativo obtener información de usuarios y grupos a través del demonio SSSD.
				\item \textbf{\texttt{libpam-sss}}: Módulo de \textit{Pluggable Authentication Modules} (PAM). Se encarga de la autenticación de usuarios de dominio durante el proceso de inicio de sesión.
				\item \textbf{\texttt{adcli}}: Herramienta esencial de línea de comandos para unir la máquina Linux a un dominio de Active Directory.
				\item \textbf{\texttt{samba-common-bin}}: Proporciona utilidades y librerías de bajo nivel necesarias para la interoperabilidad con protocolos de red de Windows (Kerberos, NTLM).
				\item \textbf{\texttt{oddjob}}: Servicio que facilita la ejecución segura de tareas privilegiadas (como \textit{root}) a solicitud de usuarios no privilegiados.
				\item \textbf{\texttt{oddjob-mkhomedir}}: Un complemento de \texttt{oddjob} que crea automáticamente el directorio personal (\texttt{/home/usuario}) para el usuario de dominio en su primer inicio de sesión.
				\item \textbf{\texttt{packagekit}}: Un conjunto de herramientas de alto nivel y un demonio para la gestión gráfica y automática de paquetes, comúnmente usado en entornos de escritorio.
			\end{itemize}
		\end{minipage}
	\end{center}
		\begin{figure}[H]
		\centering
		\begin{minipage}{0.45\textwidth}
			\centering
			\includegraphics[width=\textwidth]{Recursos/realm}
			\caption{En la terminal, \textit{realm} nos da toda la información de lo que necesitamos para "conectarnos" al dominio y casualmente, tenemos todas las dependencias previamente instaladas por lo que vamos a proceder a conectarnos al dominio }
			\label{realm}
		\end{minipage}
		\hfill
		\begin{minipage}{0.45\textwidth}
			\centering
			\includegraphics[width=\textwidth]{Recursos/ubutnudominio}
			\caption{He empleado las credenciales de administrador para unirme al domino y sé que estoy dentro pusto que en \textit{configured} aparece \textit{kerberos-member} es decir que he pasado el filtro de kerberos para autenticación y estamos dentro del dominio.}
			\label{ububntudominio}
		\end{minipage}
	\end{figure}
	\subsubsection{Comprobación de que ambas \textit{VM} se han unido al dominio del \textit{WS2022}}
		\begin{figure}[H]
				\centering
				\includegraphics[width=0.6\textwidth]{Recursos/maquinasUnidasipi}
				caption{Efectivamente en el menú de \textit{Usuarios y equopos de Active Directory} vemos que dentro del apartado de ordenadores aparecen nuestras dos máquinas virtaules. Y aunque en el guión se menciona que puede dar problemas y que hay que desactivar la búsqueda inversa, en caso de que el dns no pueda resolver el nombre del host desde la IP, como he configurado bien los punteros y los hosts, no me ha sido necesario este paso.}
		\end{figure}
		
	\subsection{\textit{Login} como administrador en las \textit{VM}s }
		\begin{figure}[H]
			\centering
			\begin{minipage}{0.45\textwidth}
				\centering
				\includegraphics[width=\textwidth]{Recursos/admintiny}
				\caption{Inicio de sesión exitosa en el dominio del \textit{AC} dentro de la máquina virutal, para entrar a esta opción es simplemente, en la pantalla de inicio de sesión darle a otro usuario y teclear las credenciales de Administrador.}
				\label{admintiny}
			\end{minipage}
			\hfill
			\begin{minipage}{0.45\textwidth}
				\centering
				\includegraphics[width=\textwidth]{Recursos/adminUbuntu}
				\caption{Como he unido bien la máquina ubuntu al AD, ahora puedo iniciar sesión como administrador, pero ojo, no tengo directorio \textit{/home}}
				\label{adminububntudominio}
			\end{minipage}
		\end{figure}
		\subsubsection{Asignación del directorio \textit{HOME} para la máquina Ubuntu}
				\begin{center}
				\begin{minipage}{0.4\textwidth}
					\centering
					\includegraphics[width=\textwidth]{Recursos/pam}
					\captionof{figure}{Tras actualizar el sistema, vemos que tenemos conexión a internet y me instalo todos los paquetes que menciona el guión.}
					\label{ModificacionPAM}
				\end{minipage}
				\hfill
				\begin{minipage}{0.55\textwidth}
					\textit{PAM (Pluggable Authentication Modules}) es un marco de trabajo de bajo nivel en Linux que separa los programas de la autenticación real. Cuando un programa (como\textit{ login, sudo o sshd}) necesita autenticar a un usuario, simplemente llama a la biblioteca PAM y consulta los archivos de configuración (\textit{/etc/pam.d/*}) para determinar qué módulos debe ejecutar.\\
					En concreto al añadir la línea \textit{session optional pam\_mkhomedir.so skel=/etc/skel umask=077 } lo que estoy haciendo es: crear el directio\textit{ /home} copiando los ficheros el directorio \textit{/etc/skel} dándole el permisos 700 (rwx- --- - ---) para que sólo el usuario pueda acceder a ese home  (para eso emplea el módulo \textit{mkhomdir.so}). Además es importante colocarlo después del módulo \textit{pam\_sss.so} porque este es el encargado de confirmar la identidad del usuario dentro del \textit{AD}.
				\end{minipage}
			\end{center}
		\subsubsection{\textit{Login} en el dominio con un \textit{/HOME} creado}
			\begin{figure}[H]
				\centering
				\includegraphics[width=\textwidth]{Recursos/homeCreado}
				\caption{Efectivamente ahora hay un directorio home para el Administrador dentro del dominio en la máquina.}
			\end{figure}
\section{Comentarios finales}
	Aunque los usos del \textit{AD} son muy amplios, en concreto en esta práctica me he centrado en el uso de un autenticador único para poder realizar el inicio de sesión en diferentes máquinas, facilitando la adminitración de los pcs dentro de un dominio. Para esto tenemos que tener en cuenta que hemos aplicado un sistema de autenticación único, con un DNS (en ambas direcciones) para poder resolver los hosts con nombre y con IP, además de virtualizar empleando la tecología \textit{Hyper-V} de una manera muy simple y sencilla, crear conmutadores de red para estas máquinas virtuales y permitir tanto la conexión a internet como la comunicación entre ellas, la primera de las tareas empleando el servidor como orquestador de las redes y las máquinas de tal manera que todo el tráfico pasa por él (tiene sentido puesto que el DNS está instalado en nuestro servidor añadiéndole una capa de seguridad puesto que manejamos direcciones privadas y públicas en el espacio de direcciones.)
	\newpage
	\addcontentsline{toc}{section}{Bibliografía}
	\nocite{*}
	\printbibliography
\end{document}

%https://learn.microsoft.com/en-us/openspecs/windows_protocols/ms-rdpbcgr/5073f4ed-1e93-45e1-b039-6e30c385867c
%https://www.palentino.es/pdfs/Windows-Server-2022-Libro-Castellano-by-palentino.pdf
%https://learn.microsoft.com/en-us/windows-server/
%https://learn.microsoft.com/en-us/windows-server/virtualization/hyper-v/overview
%https://community.nethserver.org/t/active-directory-couldnt-join-realm-insufficient-permissions-to-join-the-domain/24672
%https://learn.microsoft.com/en-us/windows-server/identity/ad-ds/manage-user-accounts-in-windows-server